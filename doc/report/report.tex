% Copyright (C) 2020-2021 Diogo Rodrigues, Breno Pimentel
% Distributed under the terms of the GNU General Public License, version 3

\documentclass[runningheads]{llncs}

\usepackage{graphicx}
% Used for displaying a sample figure. If possible, figure files should
% be included in EPS format.
%
% If you use the hyperref package, please uncomment the following line
% to display URLs in blue roman font according to Springer's eBook style:
% \renewcommand\UrlFont{\color{blue}\rmfamily}

\usepackage[binary-units=true]{siunitx} %SI units

\begin{document}
\title{TuCoTuTeCo}
\subtitle{Allocating students to classes during COVID-19}
%
%\titlerunning{Abbreviated paper title}
% If the paper title is too long for the running head, you can set
% an abbreviated paper title here
%
\author{Diogo Miguel Ferreira Rodrigues (\email{diogo.rodrigues@fe.up.pt})\inst{1},\\
Breno Accioly de Barros Pimentel (\email{up201800170@fe.up.pt})\inst{1}}

\authorrunning{D. Rodrigues and B. Pimentel}

\institute{
Faculty of Engineering -- University of Porto, Portugal\\
PLOG -- Class 3MIEIC02, Group TuCoTuTeCo\_4
}
%
\maketitle              % typeset the header of the contribution
%
\begin{abstract}
The abstract should briefly summarize the contents of the paper in
150--250 words.

\keywords{
First keyword \and
Second keyword \and
Another keyword.}
\end{abstract}

\section{Introduction}

Student--class allocation is a routine problem for educational institutions, and is usually solved as a constrainted resource allocation problem where classes are the resources, the agents are the students and subject to constraints such as a student not having overlapping classes of different subjects and balancing class sizes. However, as a result of the COVID-19 pandemic, further constraints must be applied to increase physical distancing and reduce the number of contacts, such as more strongly enforcing class balancing, and also balancing the number of students with even and odd ID number in a class so students with even and odd IDs can have presential classes in alternating weeks.

We will thus model this problem as a constraint optimization problem (COP) with a set of soft and hard constraints, and provide it with a value function that should be optimized. We encoded the problem as a Prolog program using SICStus Prolog and the constraint logic programming (CLP) module \texttt{clpfd} (\textit{CLP over Finite Domains}), with an interface to solve an instance of the problem. We then generate a set of different scenarios with varying dimensions to analyse the time complexity of the solution. Finally we present the results of our experiments.

\section{Problem description}

\section{Approach}

Our priority will be to take a logic programming approach right from the start, and only when concepts become more complex will we use more mathematical notations.

\subsection{Data}

The input data is made of:
\begin{enumerate}
    \item Facts using the predicate \texttt{schedule(Subject, Class, Time)}, which means class \texttt{Class} of subject \texttt{Subject} is attending a class at time \texttt{Time}. To simplify without loss of generality, subjects are numbered sequentially starting in 1, classes of a subject are numbered sequentially starting in 1, and time is an integer representing a time slot (for instance, time can be divided into $\SI{30}{\minute}$ slots where the first slot is on Monday from 00:00 to 00:30). This representation of time is merely a convenient way to understand time, as the only thing that matters is the boolean relation that states if two subject-class pairs overlap.
    \item Facts using the predicate \texttt{student(ID, Order, Subject, Class)}, which means student with ID \texttt{ID} has chosen class \texttt{Class} of subject \texttt{Subject} as its \texttt{Order}-th choice. \texttt{Order} is numbered sequentially and each student must have between 3 and 6 schedule options (inclusive).
    \item Facts using the predicate \texttt{grade(ID, Grade)}, which means student with ID \texttt{ID} has average grade \texttt{Grade}.
\end{enumerate}

\subsection{Decision variables}

\subsection{Constraints}

\subsection{Evaluation function}

\section{Solution presentation}

\section{Experiments and results}

\subsection{Dimensional analysis}

\subsection{Search strategies}

\section{Conclusions and future work}

%
% ---- Bibliography ----
%
% BibTeX users should specify bibliography style 'splncs04'.
% References will then be sorted and formatted in the correct style.

\bibliographystyle{llncs2e/splncs04}
\bibliography{report}

\end{document}
